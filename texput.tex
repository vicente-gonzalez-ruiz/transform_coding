% Emacs, this is -*-latex-*-
\title{Transform Coding}

\maketitle
\tableofcontents

\section{Information concentration}
%{{{

\href{https://web.stanford.edu/class/ee398a/handouts/lectures/07-TransformCoding.pdf}{Transform
  coding} can exploit
\href{https://en.wikipedia.org/wiki/Correlation_and_dependence}{correlation}
in \href{https://en.wikipedia.org/wiki/Signal}{signals} to concentrate
its
\href{https://en.wikipedia.org/wiki/Information}{information}\footnote{That
can be estimated through the
\href{https://en.wikipedia.org/wiki/Variance}{variance} or the
\href{https://en.wikipedia.org/wiki/Entropy}{entropy}} in a subset of
transformed elements called \emph{coefficients}, by decorrelating the
input samples~\cite{sayood2017introduction}. Normally, after the
transformation, quantization of the signal is more
effective\footnote{for the same bit-rate, the lossy compression ratios
are higher.} when the energy of the signal is accumulated in an small
number of coefficients because we can dedicate more bits to encode the
more energetic ones.

%}}}

\section{Transform Coding (TC) versus Vector Quantization (VQ)}
%{{{

Both, TC and VQ works exploiting the correlation between samples,
although SQ (Scalar Quantization) does not. Therefore, we can expect
that the RD performance~\cite{vruiz__information_theory} of a
(TC+SQ)-based codec should perform in the RD domain similarly to VQ.

%}}}

\section{Matrix form of the transform}
%{{{

All linear\footnote{Non-linear transform are also possible, but their
mathematical treatment is different.} transforms can be
described as a
\href{https://en.wikipedia.org/wiki/Matrix_multiplication}{matrix-vector
  product}~\cite{strang4linear}
\begin{equation}
  \mathbf{y} = \mathbf{K}\mathbf{x},
  \label{eq:forward_transform_matrix_form}
\end{equation}
where $\mathbf{x}$ is the input signal, $\mathbf{K}$ is the analysis
transform matrix, and $\mathbf{y}$ is the output decomposition. The
coefficients are found by
\begin{equation}
  {\mathbf{y}}_i = \langle {\mathbf{K}}_i, {\mathbf{x}}_i\rangle,
\end{equation}
where ${\mathbf{K}}_i$ is the $i$-th row of $\mathbf{K}$, and
$\langle\cdot,\cdot\rangle$ denotes the
\href{https://mathworld.wolfram.com/InnerProduct.html}{inner
  product}. This basically means that ${\mathbf{y}}_i$ is proportional to the
\href{https://en.wikipedia.org/wiki/Similarity_(geometry)}{similarity}
between the input signal $\mathbf{x}$ and the
\href{https://en.wikipedia.org/wiki/Finite_impulse_response}{taps} of
the \href{https://en.wikipedia.org/wiki/Digital_filter}{filter}
${\mathbf{K}}_i$.\footnote{These
\href{https://cseweb.ucsd.edu/classes/fa17/cse166-a/lec13.pdf}{slides}
can help you with this key idea.} The inverse (synthesis) transform is
computed by
\begin{equation}
  \mathbf{x} = {\mathbf{K}}^{-1}\mathbf{y},
  \label{eq:backward_transform_matrix_form}
\end{equation}
where ${\mathbf{K}}^{-1}$ denotes to the inverse matrix of
$\mathbf{K}$. When ${\mathbf K}$ is orthonormal, it holds that
\begin{equation}
  \mathbf{K}={\mathbf{K}}^{-1}={\mathbf{K}}^{\text T},
  \label{eq:orthogonal_matrix}
\end{equation}
where ${\mathbf{K}}^{\text T}$ represents the transpose matrix of
$\mathbf{K}$. Without considering scale factors,
Eq.~\ref{eq:orthogonal_matrix} is also true for all
\href{https://en.wikipedia.org/wiki/Orthogonality}{orthogonal}
transforms. Orthogonal and orthonormal transforms satisfy that
\begin{equation}
  \langle {\mathbf{K}}_i, {\mathbf{K}}_j\rangle = 0, \forall i\neq j.
\end{equation}

%}}}

\section{Transform coding gain}
%{{{

Transforms are used in signal coding to provide relative (between
subbands) energy compaction. The capatility of a transform to achieve
this effect can be estimated by the so called \emph{transform coding
gain}~\cite{vetterli1995wavelets,sayood2017introduction} defined by
\begin{equation}
G = \frac{\frac{1}{N}\sum_{n=1}^N{\sigma_n^2}}{(\prod_{n=1}^N\sigma_n^2)^{\frac{1}{N}}},
\end{equation}
where $N$ is the number of coefficients in a block (in our case, the
number of coefficients in a transformed pixel, i.e., $N=3$), and
$\sigma_n^2$ is the variance of the $n$-th coefficient in the
block. As it can be seen, $G$ is the ratio of the arithmetic mean of
the variances of the transform coefficients to their geometric
mean. Notice that $G$ is computed inside of a block (a pixel in the
case of a color transform), not among blocks (pixels).

%}}}

\subsection{Block-based transform coding}
%{{{

Some transforms, such as the DCT are applied by 2D blocks which (for
example, of $8\times 8$ pixels). This a direct consequence of that,
usually, the transform losses compaction efficiency when the block size is
increased (although this depends on the signal characteristics). When
the coefficients of several blocks are considered together, they form
a \href{https://en.wikipedia.org/wiki/Sub-band_coding}{subband}, and
the collection of subbands, a
\href{https://en.wikipedia.org/wiki/Discrete_wavelet_transform}{decomposition}~\cite{vetterli2014foundations},
and the
\href{https://en.wikipedia.org/wiki/Array_data_structure#Element_identifier_and_addressing_formulas}{index}
of the subband is related to the
\href{https://en.wikipedia.org/wiki/Frequency}{frequency} of the
signal. For example, in the case of the
\href{https://en.wikipedia.org/wiki/Digital_image}{images}, the
position of the
\href{https://en.wikipedia.org/wiki/Coefficient}{coefficients} in the
subbands is related to
\href{https://github.com/Sistemas-Multimedia/Sistemas-Multimedia.github.io/blob/master/milestones/07-DCT/block_DCT_compression.ipynb}{the
  spatial area where the corresponding pixels are found}.

%}}}


\section{Resources}
\bibliography{maths,data-compression,signal-processing,DWT,image-compression,image-processing,information_theory}

